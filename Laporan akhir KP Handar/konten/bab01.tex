%%%%%%%%%%%%%%%%%%%%%%%%%%%%%%%%%%%%%%%%%%%%%%%%%%%%%%%%%%%%%%%%%%%%%%
% BAB PENDAHULUAN:
%=====================================================================
\pagenumbering{arabic}
\renewcommand{\thechapter}{\Roman{chapter}}
\addtocontents{toc}{\vskip10pt}
\chapter{PENDAHULUAN}
\renewcommand{\thechapter}{\arabic{chapter}}
\pagestyle{konten}
%---------------------------------------------------------------------

%=====================================================================
\section{Latar Belakang}
%=====================================================================

Stanene, material dua dimensi berbasis timah (Sn), telah menjadi subjek penelitian yang signifikan karena sifat elektroniknya yang unik dan potensial. Sebagai anggota dari keluarga material dua dimensi, stanene menawarkan fitur struktural dan elektronik yang menarik yang bisa memainkan peran penting dalam pengembangan teknologi masa depan. Salah satu sifat yang paling menarik dari stanene adalah potensinya sebagai isolator topologi, sebuah material yang memiliki sifat elektronik yang unik di mana ia bertindak sebagai insulator di bagian dalam namun konduktor pada permukaannya. Ini membuatnya sangat relevan untuk aplikasi dalam elektronik spintronik, di mana pengendalian dan manipulasi spin elektron menjadi kunci.

Studi tentang band gap dan struktur elektronik suatu material adalah fundamental untuk memahami bagaimana material tersebut dapat digunakan dalam aplikasi elektronik. Stanene, dengan band gap yang dapat disesuaikan, menunjukkan kemampuan untuk berfungsi dalam berbagai aplikasi elektronika, dari transistor efek medan hingga sensor berbasis semikonduktor. Keberadaan band gap ini memungkinkan kontrol yang lebih besar atas aliran elektron, yang penting untuk kinerja perangkat elektronik yang efisien. Selain itu, karakteristik topologi stanene menawarkan ketahanan terhadap gangguan eksternal, yang dapat meningkatkan stabilitas dan keandalan perangkat elektronik yang menggunakan material ini.

Metode Density Functional Theory (DFT) memainkan peran penting dalam mengeksplorasi sifat elektronik stanene. DFT adalah alat komputasi yang sangat kuat yang digunakan untuk mempelajari struktur elektronik material dengan sangat detail, memungkinkan peneliti untuk mengevaluasi bagaimana elektron berinteraksi dalam material tersebut. Dalam konteks stanene, DFT memungkinkan penghitungan dan visualisasi distribusi densitas elektron, band structure, dan fitur elektronik lainnya yang mendasari sifat topologisnya. 

Penelitian ini bertujuan untuk menyelidiki sifat elektronik stanene dengan fokus khusus pada karakteristik struktural dan topologisnya menggunakan DFT. Dengan mempelajari distribusi densitas elektron dan band structure stanene, penelitian ini diharapkan dapat mengungkapkan bagaimana properti ini dapat dimanfaatkan dalam teknologi elektronika dan spintronika.

Dengan pemahaman yang lebih dalam tentang sifat elektronik stanene melalui pendekatan DFT, penelitian ini bertujuan untuk membuka jalan bagi pengembangan perangkat elektronik yang lebih efisien dan inovatif. Pengetahuan ini tidak hanya penting untuk aplikasi praktis, tetapi juga memberikan wawasan fundamental dalam fisika zat mampat, membantu memperdalam pemahaman kita tentang sifat dan potensi material dua dimensi.


%=====================================================================
\section{Rumusan Masalah}
%=====================================================================

Berdasarkan latar belakang yang telah disebutkan, penulis melakukan analsis kinerja dengan permasalahan yang ditemukan sebagai berikut.
\setlist{nolistsep}
\begin{enumerate}[noitemsep]
      \item Bagaimana cara mengetahui karakteristik material stanene dengan metode DFT?
      \item Bagaimana struktur geometri dari stanene?
      \item Bagaimana struktur elektronik dari stanene?
\end{enumerate}

%=====================================================================
\section{Tujuan}
%=====================================================================

Adapun tujuan yang ingin dicapai dalam penelitian ini adalah
\setlist{nolistsep}
\begin{enumerate}[noitemsep]
    \item  Mengetahui karakteristik material stanene dengan metode DFT
    \item  Mengetahui struktur geometri dari stanene
    \item  Mengetahui struktur elektronik dari stanene
\end{enumerate}


%===================================================================
\section{Batasan Masalah}
%===================================================================

Batasan masalah dalam penelitian ini adalah:
\setlist{nolistsep}
\begin{enumerate}[noitemsep]
    \item Hanya terdiri dari satu \textit{layer} stanene
    
\end{enumerate}


%=====================================================================
\section{Manfaat}
%=====================================================================
Manfaat yang diharapkan penulis dengan dilakukannya penelitian ini adalah
\setlist{nolistsep}
\begin{enumerate}[noitemsep]
    \item Memberikan rujukan dalam pengembangan material semikonduktor \textit{stanene}.
    \item Memperoleh pengetahuan tentang \textit{band gap engineering}.
    \item Memperoleh pengetahuan tentang karakteristik elektornik dari material \textit{stanene}
\end{enumerate}


%%%%%%%%%%%%%%%%%%%%%%%%%%%%%%%%%%%%%%%%%%%%%%%%%%%%%%%%%%%%%%%%%%%%%%