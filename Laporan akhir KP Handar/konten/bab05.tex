%%%%%%%%%%%%%%%%%%%%%%%%%%%%%%%%%%%%%%%%%%%%%%%%%%%%%%%%%%%%%%%%%%%%%%
%%%%%%%%%%%%%%%%%%%%%%%%%%%%%%%%%%%%%%%%%%%%%%%%%%%%%%%%%%%%%%%%%%%%%%
% BAB PENUTUP
%=====================================================================
\renewcommand{\thechapter}{\Roman{chapter}}
\addtocontents{toc}{\vskip10pt}
\chapter{PENUTUP}
\renewcommand{\thechapter}{\arabic{chapter}}
%---------------------------------------------------------------------

%=====================================================================
\section{Kesimpulan}
%=====================================================================

Berdasarkan penelitian yang telah dilakukan, dapat ditarik kesimpulan bahwa stanene bersifat konduktor yang berpusat pada titik \texttt{k-points} K. Perbedaan nilai dari penelitian sebelumnya yang dilakukan oleh \citeauthor{sagar2019} kemungkinan besar disebabkan oleh perbedaan parameter atau kesalahan kalkulasi yang digunakan. Dalam penelitian sebelumnya juga disebutkan bahwa untuk struktur 2-dimensi stanene dilakukan dengan menggunakan fungsional PBE-GGA yang diverifikasi oleh fungsional LDA tetapi pada penelitian ini hanya menggunakan fungsional PBE-GGA

%=====================================================================
\section{Saran}
%=====================================================================
\begin{enumerate}
    \item Untuk penelitian selanjutnya, disarankan menggunakan parameter yang digunakan
    \item Mengecek parameter dengan paper yang telah ada
    \item Mengecek hasil sudah akurat atau belum

\end{enumerate}

