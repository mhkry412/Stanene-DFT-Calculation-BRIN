%%%%%%%%%%%%%%%%%%%%%%%%%%%%%%%%%%%%%%%%%%%%%%%%%%%%%%%%%%%%%%%%%%%%%%
%
%   Abstract
%
%%%%%%%%%%%%%%%%%%%%%%%%%%%%%%%%%%%%%%%%%%%%%%%%%%%%%%%%%%%%%%%%%%%%%%

\begin{center}
    \addcontentsline{toc}{chapter}{\textit{ABSTRACT}}
    \pagestyle{fancy}
\end{center}

%---------------------------------------------------------------------

\begin{center}
    {\textbf{\MakeUppercase{\judulTAInggris}}}
\end{center}

\vspace{5mm}

\noindent \begin{tabular}{l c l}
    \textbf{Name}       & \textbf{:} & \textbf{\namaMahasiswa}  \\[-1mm]
    \textbf{NRP}        & \textbf{:} & \textbf{\noIndukMahasiswa}  \\[-1mm]
    \textbf{Department} & \textbf{:} & \textbf{\namaDepartemenInggris}  \\[-1mm]
    \textbf{Supervisors}& \textbf{:} & \textbf{1. \namaDosenPembimbingSatu}  \\[-1mm]
                        &            & \textbf{2. \namaDosenPembimbingDua}
\end{tabular}

%---------------------------------------------------------------------

\vspace{5mm}

\begin{center}
    \noindent {\textbf{{\textit{Abstract}}}}
\end{center}

%---------------------------------------------------------------------

% Catatan: Gunakan \singlespacing di tiap awal paragraf

{\singlespacing\indent% 
\textit{}
This study focuses on analyzing the electronic properties of stanene through band structure and density of states calculations using the Density Functional Theory (DFT) method. The calculation process involves optimizing the structure and verifying the convergence of parameters such as kinetic energy cutoff and K-points. The analysis results indicate that stanene exhibits conductor properties centered at the \textit{K} point. This study also identifies differences in results compared to previous research, likely due to variations in calculation parameters. In this research, the tetrahedron method was used to calculate the density of states as it provides higher accuracy compared to other methods such as Gaussian. The charge density distribution was analyzed to understand the bonding properties between atoms in stanene, showing significant electron delocalization, which illustrates limited charge transport capability. The conclusions of this study support the use of stanene in future electronic applications, although further verification of the parameters used is necessary to ensure the accuracy of the results.}

%---------------------------------------------------------------------

\vspace{5mm}

\noindent \textbf{Keywords: Band Structure, Density of States, DFT, Stanene} \textit{} % Kata kunci dalam bahasa Inggris

%%%%%%%%%%%%%%%%%%%%%%%%%%%%%%%%%%%%%%%%%%%%%%%%%%%%%%%%%%%%%%%%%%%%%%