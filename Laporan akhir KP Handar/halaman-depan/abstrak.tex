%%%%%%%%%%%%%%%%%%%%%%%%%%%%%%%%%%%%%%%%%%%%%%%%%%%%%%%%%%%%%%%%%%%%%%
%
%   Abstrak
%
%%%%%%%%%%%%%%%%%%%%%%%%%%%%%%%%%%%%%%%%%%%%%%%%%%%%%%%%%%%%%%%%%%%%%%

\begin{center}
    \addcontentsline{toc}{chapter}{ABSTRAK}
    \pagestyle{fancy}
\end{center}

%---------------------------------------------------------------------

\begin{center}
    {\textbf{\MakeUppercase{\judulTA}}}
\end{center}

\vspace{5mm}

\noindent \begin{tabular}{l c l}
    \textbf{Nama}       & \textbf{:} & \textbf{\namaMahasiswa}  \\[-1mm]
    \textbf{NRP}        & \textbf{:} & \textbf{\noIndukMahasiswa}  \\[-1mm]
    \textbf{Departemen} & \textbf{:} & \textbf{\namaDepartemen}  \\[-1mm]
    \textbf{Pembimbing} & \textbf{:} & \textbf{1. \namaDosenPembimbingSatu}  \\[-1mm]
                        &            & \textbf{2. \namaDosenPembimbingDua}
\end{tabular}

%---------------------------------------------------------------------

\vspace{5mm}

\begin{center}
    \noindent {\textbf{{Abstrak}}}
\end{center}

%---------------------------------------------------------------------

% Catatan: Gunakan \singlespacing di tiap awal paragraf

{\singlespacing\indent%
Penelitian ini berfokus pada analisis sifat elektronik stanene melalui kalkulasi struktur pita dan rapat keadaan menggunakan metode Density Functional Theory (DFT). Proses kalkulasi melibatkan optimasi struktur dan verifikasi konvergensi parameter seperti cutoff energi kinetik dan titik K-points. Hasil analisis menunjukkan bahwa stanene memiliki sifat konduktor yang berpusat pada titik \textit{K}. Penelitian ini juga mengidentifikasi perbedaan hasil dibandingkan penelitian sebelumnya yang kemungkinan besar disebabkan oleh perbedaan parameter kalkulasi. Dalam penelitian ini, metode tetrahedra digunakan untuk menghitung rapat keadaan karena memberikan akurasi yang lebih tinggi dibandingkan metode lain seperti gaussian. Distribusi kerapatan muatan dianalisis untuk memahami sifat ikatan antara atom dalam stanene, menunjukkan adanya delokalisasi elektron yang signifikan, yang menggambarkan kemampuan transportasi muatan terbatas. Kesimpulan penelitian ini mendukung penggunaan stanene dalam aplikasi elektronik masa depan, meskipun masih diperlukan verifikasi lebih lanjut terhadap parameter yang digunakan untuk memastikan keakuratan hasil.
}

%---------------------------------------------------------------------

\vspace{5mm}

\noindent \textbf{Kata kunci: DFT, Rapat Keadaan, Struktur Pita, Stanene} \textit{} % Kata kunci dalam bahasa Indonesia

%%%%%%%%%%%%%%%%%%%%%%%%%%%%%%%%%%%%%%%%%%%%%%%%%%%%%%%%%%%%%%%%%%%%%%