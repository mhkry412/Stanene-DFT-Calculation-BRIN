%%%%%%%%%%%%%%%%%%%%%%%%%%%%%%%%%%%%%%%%%%%%%%%%%%%%%%%%%%%%%%%%%%%%%%
% FORMAT DAN PENGATURAN UMUM:
%=====================================================================
\usepackage{multirow}
\usepackage[table]{xcolor}
\usepackage{tabularx,tikz}
\usepackage{caption}
\usepackage{float}
\usepackage{listings}
\usepackage{subcaption}
\usepackage{graphicx}
\usepackage{enumitem}
\usepackage{amsmath,amssymb,charter,color}
\usepackage[hidelinks]{hyperref}
\usepackage[utf8]{inputenc}
\usepackage{lipsum} % hanya untuk membuat teks dummy
\usepackage{wrapfig}
\usepackage{emptypage}
\usepackage{setspace}
\usepackage{rotating}



%======================================================================
%   Mengatur margin 
%======================================================================
\usepackage[top=3cm,left=3cm,right=2cm,bottom=2.5cm]{geometry}
    \linespread{1.3} % Gunakan 1.3 untuk spasi satu setengah, 
                     % Gunakan 1.6 untukspasi dua
    
%======================================================================
%   Mengatur bahasa yang digunakan
%======================================================================

\usepackage[bahasa]{babel}
    \selectlanguage{bahasa}

%======================================================================
%   Mengatur format BAB, SUBBAB, Floating Gambar dan Tabel
%======================================================================

\usepackage[explicit]{titlesec}
    \titleformat{\chapter}[display]
        {\normalfont\bfseries\centering}
        {\large\MakeUppercase{\chaptername}~\thechapter}
        {0.5em}{\large \MakeUppercase{#1}}
    \titleformat{\section}
        {\normalfont\normalsize\bfseries}{\thesection}
        {0.5em}{#1}
    \titleformat{\subsection}
        {\normalfont\normalsize\bfseries}{\thesubsection}
        {0.5em}{#1}
    
\usepackage{tocloft,etoolbox}
\apptocmd{\appendix}
    {\addtocontents{toc}{  
    \protect\addtolength\protect\cftchapnumwidth{-\mylength}
    \protect\renewcommand{\protect\cftchappresnum}{LAMPIRAN~}
    \protect\settowidth\mylength{
    \bfseries\protect\cftchappresnum\protect\cftchapaftersnum}
    \protect\addtolength\protect\cftchapnumwidth{\mylength}}}{}{}
\newlength\mylength

\renewcommand\cftchappresnum{BAB~}
\settowidth\mylength{\bfseries\cftchappresnum\cftchapaftersnum}
\addtolength\cftchapnumwidth{\mylength}
\renewcommand{\cftdotsep}{1}
\renewcommand{\cftchapleader}{\cftdotfill{\cftsecdotsep}}

\renewcommand\cftfigpresnum{Gambar~}
\settowidth\mylength{\cftfigpresnum\cftfigaftersnum}
\addtolength\cftfignumwidth{\mylength}

\renewcommand\cfttabpresnum{Tabel~}
\settowidth\mylength{\cfttabpresnum\cfttabaftersnum}
\addtolength\cfttabnumwidth{\mylength}

%======================================================================
%   Mengatur background pada cover
%======================================================================
\usepackage[pages=some]{background}
\backgroundsetup{
    scale=1,
    color=black,
    opacity=1.0,
    angle=0,
    contents={%
    \includegraphics[width=\paperwidth,height=\paperheight]
    {./gambar/cover1-01.png}
    }%
}

%======================================================================
%   Mengatur jenis font yang digunakan
%======================================================================
\usepackage{fontspec}
\setmainfont{Times New Roman} 
\setsansfont{Trebuchet MS} 
\setmonofont{Inconsolata}

%======================================================================
%   Untuk mendefinisikan halaman kosong
%======================================================================
\newcommand\halamanKosong{
    \newpage
    \vspace*{\fill}
    \begin{center}
        \textit{Halaman ini sengaja dikosongkan}
    \end{center}
    \vspace{\fill}
    \clearpage
}

%======================================================================
%   Mengatur agar paragraf pertama mempunyai indentasi
%======================================================================
\usepackage{indentfirst}
\setlength{\parindent}{2em} 

%======================================================================
%   Mengatur tentang jarak antara judul section dan paragraf
%======================================================================
\usepackage{titlesec}

\titlespacing\section{0pt}{12pt plus 4pt minus 2pt}{0pt plus 2pt minus 2pt}
\titlespacing\subsection{0pt}{12pt plus 4pt minus 2pt}{0pt plus 2pt minus 2pt}
\titlespacing\subsubsection{0pt}{12pt plus 4pt minus 2pt}{0pt plus 2pt minus 2pt}

%======================================================================
%   Mengatur header dan footer
%======================================================================
\usepackage{fancyhdr}
    \fancyhead{}
    \fancyfoot{}
    \setlength{\headheight}{15pt}
    \setlength{\headsep}{12pt}
    \setlength{\footskip}{30pt}
    \renewcommand{\headrulewidth}{0pt}
    \renewcommand{\footrulewidth}{0pt}
    
    \fancypagestyle{romawi}{%
    \setlength{\headheight}{15pt}
    \setlength{\headsep}{12pt}
    \setlength{\footskip}{30pt}
    \fancyfoot[CE,CO]{\thepage}
    \renewcommand{\headrulewidth}{0pt}
    \renewcommand{\footrulewidth}{0pt}
    }
    
    \fancypagestyle{konten}{%
    \setlength{\headheight}{15pt}
    \setlength{\headsep}{12pt}
    \setlength{\footskip}{30pt}
    \fancyhead[LE,RO]{\thepage}
    \fancyfoot[CE,CO]{}
    \renewcommand{\headrulewidth}{0pt}
    \renewcommand{\footrulewidth}{0pt}
    }
    
%======================================================================
%   Mengatur tentang format referensi
%======================================================================

\usepackage[square]{natbib}

%======================================================================
%   Mengatur template untuk menulis code
%======================================================================

\usepackage{listings}
\usepackage[framemethod=default]{mdframed}

\newmdenv[innerlinewidth=0.5pt, 
roundcorner=4pt,
linecolor=red,
innerleftmargin=6pt,
innerrightmargin=6pt,
innertopmargin=6pt,
innerbottommargin=6pt,
backgroundcolor=red,
]{mybox}
\newcommand{\unitcell}{\textit{unit cell }}
\newcommand{\exchange}{\textit{exchange }}
\newcommand{\corr}{\textit{correlation }}
\newcommand{\eh}[1]{{\color{red} EH:{#1}}}
\newcommand{\schro}{Schr{\"o}dinger }
\newcommand{\be}{\begin{equation}}
\newcommand{\ee}{\end{equation}}
\newcommand{\bea}{\begin{eqnarray}}
\newcommand{\eea}{\end{eqnarray}}
\newcommand{\HH}{{\cal H}}
\newcommand{\RR}{{\cal R}}
\newcommand{\p}{\partial}
\newcommand{\s}{\sigma}
\newcommand{\la}{\langle}
\newcommand{\ra}{\rangle}
\newcommand{\lla}{\left\langle}
\newcommand{\rra}{\right\rangle}
\newcommand{\lb}{\left[}
\newcommand{\rb}{\right]}
\newcommand{\lp}{\left(}
\newcommand{\rp}{\right)}
\newcommand{\Tr}{{\rm \, Tr\,}}
\newcommand{\bra}[1]{\la #1|}
\newcommand{\ket}[1]{| #1\ra}
\newcommand{\sgn}{{\rm sgn}\,}
\renewcommand{\Im}{{\rm Im}\,}
\renewcommand{\Re}{{\rm Re}\,}
\renewcommand{\vec}[1]{{\bf #1}}
\newcommand{\eps}{\varepsilon}
\renewcommand{\tilde}{\widetilde}
\def\nn{\nonumber\\}

\definecolor{codegreen}{rgb}{0,0.6,0}
\definecolor{codegray}{rgb}{0.5,0.5,0.5}
\definecolor{codepurple}{rgb}{0.58,0,0.82}
\definecolor{backcolour}{rgb}{0.95,0.95,0.92}

\lstdefinestyle{mystyle}{
    backgroundcolor=\color{backcolour},   
    commentstyle=\color{codegreen},
    keywordstyle=\color{magenta},
    numberstyle=\tiny\color{codegray},
    stringstyle=\color{codepurple},
    basicstyle=\ttfamily\small,
    breakatwhitespace=false,         
    breaklines=true,                 
    captionpos=b,                    
    keepspaces=true,                 
    numbers=left,                    
    numbersep=5pt,                  
    showspaces=false,                
    showstringspaces=false,
    showtabs=false,                  
    tabsize=2 
}

\lstset{style=mystyle}

%======================================================================
%   Agar sisi bagian kanan menjadi lebih rapih
%======================================================================
\emergencystretch=\maxdimen
\hyphenpenalty=10000
\hbadness=10000

%======================================================================
%   Mengatur jenis font untuk equations
%======================================================================
\usepackage{newtxmath}

%======================================================================
%   Mengatur format daftar isi, gambar, dan tabel
%======================================================================

\renewcommand{\cfttoctitlefont}{\hfil\large\bfseries\MakeUppercase}
\renewcommand{\cftloftitlefont}{\hfil\large\bfseries\MakeUppercase}
\renewcommand{\cftlottitlefont}{\hfil\large\bfseries\MakeUppercase}
\renewcommand{\cftsecleader}{\cftdotfill{\cftdotsep}}
\setlength\cftparskip{-2pt}
\setlength\cftbeforesecskip{2pt}
\setlength\cftbeforechapskip{2pt}
\setlength\extrarowheight{5pt}

%======================================================================
%   Mengatur Diagram Alir
%======================================================================
\usepackage{tikz}
\usetikzlibrary{shapes.geometric, arrows}

\definecolor{adaee4ed-88c8-5b21-a9e7-31316ebef86f}{RGB}{255, 179, 178}
\definecolor{f3551e38-74df-57e2-b793-83d7fe876c85}{RGB}{0, 0, 0}
\definecolor{0b71a967-1f15-55a5-9bb9-70efa7b4fc58}{RGB}{51, 51, 51}
\definecolor{747aec21-333b-59ee-84e3-ddff893e5ccd}{RGB}{255, 216, 176}
\definecolor{5856d031-3da1-575c-834e-c77e9e438c62}{RGB}{162, 177, 195}

\tikzstyle{512bdd77-c3aa-5669-a956-85f7a90c6fb4} = [rectangle, rounded corners, minimum width=3cm, minimum height=1cm, text centered, font=\normalsize, color=0b71a967-1f15-55a5-9bb9-70efa7b4fc58, draw=f3551e38-74df-57e2-b793-83d7fe876c85, line width=1, fill=adaee4ed-88c8-5b21-a9e7-31316ebef86f]
\tikzstyle{69bbb168-da59-5865-902f-94e77902bf95} = [rectangle, minimum width=3cm, minimum height=1cm, text centered, font=\normalsize, color=0b71a967-1f15-55a5-9bb9-70efa7b4fc58, draw=f3551e38-74df-57e2-b793-83d7fe876c85, line width=1, fill=747aec21-333b-59ee-84e3-ddff893e5ccd]
\tikzstyle{40b3368b-3948-5ae0-af87-93343251acd0} = [rectangle, minimum width=4cm, minimum height=1cm, text centered, font=\normalsize, color=0b71a967-1f15-55a5-9bb9-70efa7b4fc58, draw=f3551e38-74df-57e2-b793-83d7fe876c85, line width=1, fill=747aec21-333b-59ee-84e3-ddff893e5ccd]
\tikzstyle{7be24b85-97d0-5b76-ba9e-d94005dca8f2} = [thick, draw=5856d031-3da1-575c-834e-c77e9e438c62, line width=2, ->, >=stealth]


\tikzstyle{startstop} = [ellipse, minimum width=3cm, minimum height=1cm, text centered, draw=black]
\tikzstyle{process} = [rectangle, minimum width=3cm, minimum height=1cm, text centered, draw=black]
\tikzstyle{arrow} = [thick,->,>=stealth]
%%%%%%%%%%%%%%%%%%%%%%%%%%%%%%%%%%%%%%%%%%%%%%%%%%%%%%%%%%%%%%%%%%%%%%
